% vim: set filetype=tex :
% !TEX program = xelatex
%-------------------------
% Resume in Latex
% Based off of: https://github.com/sb2nov/resume
% License : MIT
%------------------------

\documentclass[letterpaper,10pt]{ctexart}

\usepackage{latexsym}
\usepackage{fontawesome5}
\usepackage[empty]{fullpage}
\usepackage{titlesec}
\usepackage{marvosym}
\usepackage[usenames,dvipsnames]{color}
\usepackage{verbatim}
\usepackage{enumitem}
\usepackage[hidelinks]{hyperref}
\usepackage{fancyhdr}
\usepackage{tabularx}
\usepackage[super]{nth}
\usepackage{booktabs}
\usepackage{accsupp}
\usepackage{xifthen}

%----------FONT OPTIONS----------
% Set Chinese fonts for macOS
% Using system Chinese fonts available on macOS
\setCJKmainfont{STSong}[BoldFont=STHeiti,ItalicFont=STKaiti]
% Alternative fonts if above don't work:
% \setCJKmainfont{Songti SC}[BoldFont=Heiti SC,ItalicFont=Kaiti SC]
\linespread{1.0}

\pagestyle{fancy}
\fancyhf{} % clear all header and footer fields
\fancyfoot{}
\fancyhead{}
\renewcommand{\headrulewidth}{0pt}
\renewcommand{\footrulewidth}{0pt}

% Adjust margins
\addtolength{\oddsidemargin}{-0.5in}
\addtolength{\evensidemargin}{-0.5in}
\addtolength{\textwidth}{1in}
\addtolength{\topmargin}{-.5in}
\addtolength{\textheight}{1.0in}

\urlstyle{same}
% defines one's email
% usage: \email{<email>}
\newcommand{\email}[1]{\faEnvelope\ \href{mailto:#1}{#1}}
% defines one's phone
% usage: \phone{<phone>}
\newcommand{\phone}[1]{\faPhone\ {#1}}
% defines one's linkedin
% usage: \linkedin{<linkedin>}
\newcommand{\linkedin}[1]{\href{#1}{\faLinkedin}}
% defines one's GitHub
% usage: \github{<github>}
\newcommand{\github}[1]{\href{#1}{\faGithub}}

% defines one's homepage
% usage: \homepage{<homepage>}
\newcommand{\homepage}[2][]{\faLink\ 
  \ifthenelse{\isempty{#1}}%
    {\href{#2}{#2}}
    {\href{#2}{#1}}}
\newcommand{\researchgate}[1]{\href{#1}{\faResearchgate}}
\newcommand{\orcid}[1]{\href{#1}{\faOrcid}}

\raggedbottom
\raggedright
\setlength{\tabcolsep}{0in}

% Sections formatting
\titleformat{\section}{
  \vspace{-4pt}\scshape\raggedright\large
}{}{0em}{}[\color{black}\titlerule \vspace{-5pt}]

% Ensure that generate pdf is machine readable/ATS parsable
% \pdfgentounicode=1

%-------------------------
% Custom commands
\newcommand{\link}[2]{\href{#1}{\color{blue}\underline{#2}}}

\newcommand{\resumeItem}[1]{
  \item\small{
    {#1 \vspace{-2pt}}
  }
}

\newcommand{\resumeSubheading}[4]{
  \vspace{-2pt}\item
  \begin{tabular*}{0.97\textwidth}[t]{l@{\extracolsep{\fill}}r}
    \textbf{#1} & #2 \\
    \textit{\small#3} & \textit{\small #4} \\
  \end{tabular*}\vspace{-7pt}
}

\newcommand{\resumeSubSubheading}[2]{
  \item
  \begin{tabular*}{0.97\textwidth}{l@{\extracolsep{\fill}}r}
    \textit{\small#1} & \textit{\small #2} \\
  \end{tabular*}\vspace{-7pt}
}

\newcommand{\resumeProjectHeading}[2]{
  \item
  \begin{tabular*}{0.97\textwidth}{l@{\extracolsep{\fill}}r}
    \small#1 & #2 \\
  \end{tabular*}\vspace{-7pt}
}

\newcommand{\resumeSubItem}[1]{\resumeItem{#1}\vspace{-4pt}}

\renewcommand\labelitemii{$\vcenter{\hbox{\tiny$\bullet$}}$}

\newenvironment{resumeSubHeadingList}{\begin{itemize}[leftmargin=0.15in, label={}]}{\end{itemize}}
\newenvironment{resumeItemList}{\begin{itemize}}{\end{itemize}}
\newenvironment{resumeItemSubList}{\begin{itemize}\setlength\itemsep{0.4em}}{\end{itemize}\vspace{-5pt}}

\newcommand{\awardsTable}[1]{
	\begin{tabularx}{\textwidth}{l @{\extracolsep{\fill}} *{4}{l}}
	\emph{奖项名称}	& \emph{获奖等级}	& \emph{级别}	& \emph{颁发单位} & \emph{年份}       \\
	\midrule\midrule
	#1
	\end{tabularx}
}
\newcommand{\awardsTableRow}[6]{
	\BeginAccSupp{method=plain, ActualText=11\string\t 21}#1 & #2 & #3 & #4 & #5\EndAccSupp{} \\
}

\fancyfoot[L]{
  \small \LaTeX\ 源码: \link{https://github.com/vectorsss/cv}{github.com/vectorsss/cv} $|$ 
  网页版: \link{https://cv.zhaochi.ru/cn}{cv.zhaochi.ru/cn}
}

\renewcommand\footrule{\hrule width\textwidth}

%-------------------------------------------
%%%%%%  RESUME STARTS HERE  %%%%%%%%%%%%%%%%%%%%%%%%%%%%

\begin{document}
\begin{center}
  \begin{flushright}
    \footnotesize{\it 编译日期:\today{}}
  \end{flushright}
\end{center}
\begin{center}
  \textbf{\huge \scshape 赵驰} \\ \vspace{1pt}
  \small \email{vector.zhaochi@gmail.com} $|$ \homepage[https://chizhao.gitlab.io]{https://chizhao.gitlab.io} $|$ \linkedin{https://www.linkedin.com/in/dandanv5} $|$ \github{https://github.com/vectorsss} $|$
  \researchgate{https://www.researchgate.net/profile/Chi-Zhao-4} $|$ \orcid{https://orcid.org/0000-0002-1166-7578}
\end{center}

\section{教育经历}
\begin{resumeSubHeadingList}
  \resumeSubheading
    {圣彼得堡国立大学}{2021.09 -- 2025.06}
    {应用数学与控制过程博士}{俄罗斯,圣彼得堡}
    \resumeSubSubheading
    {论文题目:``Modeling of binary opinion dynamics in social networks of complex configurations''}{2025年4月答辩}

  \resumeSubheading
    {圣彼得堡国立大学}{2019.09 -- 2021.06}
    {应用数学与信息学硕士(GPA:4.9/5.0)}{俄罗斯,圣彼得堡}

  \resumeSubheading
    {北京理工大学}{2017.07.03 -- 2017.07.28}
    {ACM夏令营}{中国,北京}

  \resumeSubheading
    {延安大学}{2015.09 -- 2019.06}
    {计算机科学学士(GPA:3.1/4.0)}{中国,延安}
\end{resumeSubHeadingList}

\section{工作经历}
\begin{resumeSubHeadingList}

  \resumeSubheading
    {华为公司}{2021.09 -- 至今}
    {研发工程师}{俄罗斯,圣彼得堡}
  \resumeSubheading
    {江苏金智教育股份有限公司}{2016.09 -- 2019.05}
    {软件工程师}{中国,延安}

\end{resumeSubHeadingList}
\section{项目经历}
\begin{resumeSubHeadingList}
  \resumeSubheading
    {*专业模型数据治理/高价值训练数据选择(进行中)}{2025年}
    {项目负责人}{俄罗斯,圣彼得堡}
    \resumeSubSubheading{该项目专注于通过数据治理和有价值的特征/训练数据选择来提高模型精度。}{}
    \resumeSubSubheading{通过自研多样性采样算法从18亿条无线网络数据(4G和5G)中抽取了6000万条最具代表性的数据。}{}
    \resumeSubSubheading{通过自研采样和异常检测算法将模型精度提高了22\%。}{}
    \resumeSubSubheading{通过自研特征重要性算法去除了20\%的冗余特征,使训练速度提升10\%。}{}
  \resumeSubheading
    {*内部列存储算法}{2023--2024年}
    {研发工程师}{俄罗斯,圣彼得堡}
    \resumeSubSubheading{高性能、高度灵活的列存储无损压缩算法}{}
    \resumeSubSubheading{该算法将被商业化用于存储基站数据。}{}
    \resumeSubSubheading{LTE (4G) 数据 30\% 压缩率,NR (5G) 数据 40\% 压缩率,无任何性能损失。}{}
  \resumeSubheading
    {俄罗斯科学基金会资助项目 \link{https://rscf.ru/en/project/22-21-00346/}{编号:22-21-00346}}{2023年}
    {研究员}{俄罗斯,圣彼得堡}
    \resumeSubSubheading{完成了双层意见动态模型仿真,聚焦于公开意见和私密意见(EPO观点动态模型)。}{}
  \resumeSubheading
    {*内部分布式文件存储系统弹性扩展算法}{2023年}
    {研发工程师}{俄罗斯,圣彼得堡}
    \resumeSubSubheading{使用HRW哈希实现HOFS多节点无缝扩展,最小化数据移动且无性能损失。}{}
    \resumeSubSubheading{可靠性显著提升:彻底解决``因哈希环丢失而无法重新生成导致群集数据全部丢失''场景下数据访问的可靠性问题}{}
  \resumeSubheading
    {*数据包/路由数据压缩}{2023年}
    {研发工程师}{俄罗斯,圣彼得堡}
    \resumeSubSubheading{该项目通过无损压缩达到85\% 压缩率(节省7倍磁盘使用量)。}{}
  \resumeSubheading
    {*无线数据压缩}{2021--2022年}
    {研发工程师}{俄罗斯,圣彼得堡}
    \resumeSubSubheading{该项目通过有损压缩达到96\% 压缩率。}{}
  \resumeSubheading
    {* SparkSQL查询优化}{2021年}
    {研发工程师}{俄罗斯,圣彼得堡}
    \resumeSubSubheading{该项目通过谓词下推和数据结构优化将查询执行速度提升50\%。}{}
    % \resumeSubheading {大规模高维向量相似度快速计算}{2018年5月 -- 2018年8月} {算法工程师}{中国,延安}
    % \resumeSubSubheading{该项目提高了最近邻算法的效率,并获得了2018年中国软件杯三等奖。}{}

    % \resumeSubheading {基于CNN的在线新闻分类系统}{2018年3月 -- 2018年5月} {全栈工程师}{中国,延安}
    % \resumeSubSubheading{该项目获得了中国软件著作权登记证书。}{}
    % \resumeSubSubheading{该项目获得了2018年计算机设计大赛三等奖。}{}
\end{resumeSubHeadingList}

\newpage
%-----------PROJECTS-----------
\section{专利/著作权证书}
\begin{resumeSubHeadingList}
  \resumeSubheading
    {双层网络中意见动态建模程序}{2023年}
    {俄罗斯计算机软件著作权登记证书}{{\link{https://new.fips.ru/registers-doc-view/fips_servlet?DB=EVM&DocNumber=2023661532&TypeFile=html}{编号:2023661532}}}

  \resumeSubheading
    {基于卷积神经网络的在线新闻分类系统}{2018年}
    {中国计算机软件著作权登记证书}{编号:2831192}
\end{resumeSubHeadingList}

\section{开源项目}
\begin{resumeSubHeadingList}

  \resumeProjectHeading{
    \link{https://github.com/vectorsss/shapG}{\textbf{ShapG}} $|$
    \emph{python $\cdot$ 特征重要性算法 $\cdot$ 中心性度量}}{}
  \begin{resumeItemList}
    \resumeItem{基于Shapley值的特征重要性算法Python包。}
    \resumeItem{该包还可用于计算图中的中心性。}
  \end{resumeItemList}

  \resumeProjectHeading{
    \link{https://github.com/vectorsss/news_classification}{\textbf{新闻分类}} $|$
    \emph{python $\cdot$ 自然语言处理 $\cdot$ CNN $\cdot$ 文本分类}}{}
  \begin{resumeItemList}
    \resumeItem{基于卷积神经网络的在线新闻分类系统。}
    \resumeItem{该项目在2018年全国计算机设计大赛中获得省级三等奖。}
  \end{resumeItemList}
\end{resumeSubHeadingList}

\section{技能}
\begin{itemize}[leftmargin=0.15in, label={}]
  \small{\item{
        \textbf{编程语言}{:Python $\cdot$ C/C++ $\cdot$ Go $\cdot$ Rust $\cdot$ Matlab/Octave $\cdot$ Julia $\cdot$ R $\cdot$ SQL $\cdot$ \LaTeX} \\
        \textbf{机器学习库}{:Tensorflow $\cdot$  PyTorch $\cdot$ Keras $\cdot$  Scikit-Learn} \\
        \textbf{开发工具}{:Git $\cdot$ Docker $\cdot$  谷歌云平台} \\
        \textbf{操作系统}{:Windows $\cdot$ MacOS $\cdot$ Arch Linux} \\
        \textbf{语言能力}{:中文(母语),英语(流利)}
        }}
\end{itemize}

\section{研究兴趣}
\begin{resumeSubHeadingList}
  \resumeProjectHeading{
    图算法 $\cdot$ 中心性度量 $\cdot$ 机器学习
    $\cdot$ 可解释人工智能 $\cdot$ 概率论
    $\cdot$ 统计学}{}
  \resumeProjectHeading{
    数据压缩 $\cdot$ 编码理论 $\cdot$ 时间序列分析 $\cdot$
    优化 $\cdot$ 随机建模 $\cdot$ 随机过程}{}
\end{resumeSubHeadingList}
\section{教学经历}
\begin{resumeSubHeadingList}
  \resumeSubheading{R语言应用统计}{俄罗斯,圣彼得堡}{助教}{2024年}
  \resumeSubheading{统计决策与计量经济学}{俄罗斯,圣彼得堡}{助教}{2022年}
\end{resumeSubHeadingList}

\section{学术服务}
\begin{resumeSubHeadingList}
  \resumeProjectHeading{{\bf 期刊和会议审稿人}}{}
  \begin{resumeItemList}
    \resumeItem{Engineering Applications of Artificial Intelligence (EAAI)}
    \resumeItem{International Conference On Computational Optimization (ICOMP)}
  \end{resumeItemList}
\end{resumeSubHeadingList}
\section{精选论文}
\begin{enumerate}
  \fontsize{10}{10.5}\selectfont
  \item [1.] \textbf{Zhao C.}, Liu J., Parilina E. M. The Shapley Value Contribution to Explainable Artificial Intelligence: A Comprehensive Survey // {\it Dynamic Games and Applications.} -- 2025. -- Oct. -- P. 1-38. {\bf (Q2)}
  \item [2.] \textbf{Zhao C.}, Liu J., Parilina E. M. Complete-to-Sparse: A Novel Graph Construction Strategy for Efficient ShapG // {\it Mathematical Optimization Theory and Operations Research}. -- Cham : Springer Nature Switzerland. -- 2025. -- P. 180–194.
  \item [3.] \textbf{Zhao C.}, Liu J., Parilina E. M. ShapG: new feature importance method based on the Shapley value // {\it Engineering Applications of Artificial Intelligence.} -- 2025. -- May. -- Vol. 148, 110409. {\bf (Q1, IF: 8.0)}
  \item [4.] \textbf{Zhao C.}, Parilina E. M. Centrality measures and opinion dynamics in two-layer networks with replica nodes // {\it Computers and Operations Research.} -- 2026. -- Jan. -- Vol. 185, 107245. {\bf (Q1, IF: 4.3)}
  \item [5.] \textbf{Zhao C.}, Parilina E. M. Analysis of consensus time and winning rate in two-layer networks with hypocrisy of different structures // {\it Vestnik of Saint Petersburg University. Applied Mathematics. Computer Science. Control Processes.} -- 2024. -- Vol. 20, no. 2. -- P. 170-192.
  \item [6.] \textbf{Zhao C.}, Parilina E. M. Opinion Dynamics in Two-Layer Networks with Hypocrisy // {\it Journal of the Operations Research Society of China}. -- 2024. -- Mar. -- Vol. 12, no. 1. -- P. 109-132. {\bf (Q2)}
  \item [7.] \textbf{Zhao C.}, Parilina E. M. Network Structure Properties and Opinion Dynamics in Two-Layer Networks with Hypocrisy // {\it Mathematical Optimization Theory and Operations Research}. -- Cham : Springer Nature Switzerland. -- 2024. -- P. 300-314.
        \newpage
  \item [8.] \textbf{Zhao C.}, Parilina E. M. Consensus time and winning rate based on simulations in two-layer networks with hypocrisy // {\it 2023 7th Scientific School Dynamics of Complex Networks and their Applications (DCNA).} -- 2023. -- P. 68-71.
\end{enumerate}

\section{会议经历}
\begin{resumeSubHeadingList}
  \resumeSubheading{Mathematical Optimization Theory and Operations Research
                    (MOTOR 2025)}{Novosibirsk, Russia}{Oral Presentation}{July.
                    07 - 11, 2025}
  \resumeSubheading{Game Theory and Management (GTM 2025)}{Saint-Petersburg,
                    Russia}{Oral Presentation}{July. 2 - 4, 2025}
  \resumeSubheading{\nth{14} International Society of Dynamic Games (ISDG) Workshop}{Yerevan, Armenia}{Oral Presentation}{June. 11 - 13, 2025}
  \resumeSubheading{International Conference On Computational Optimization
                    (ICOMP 2024)}{Innopolis, Russia}{Visitor}{Oct. 10 - Oct.
                    12, 2024}
  \resumeSubheading{Mathematical Optimization Theory and Operations Research
                    (MOTOR 2024)}{Omsk, Russia}{Oral Presentation}{June. 30 -
                    July. 06, 2024}
  \resumeSubheading{Game Theory and Management (GTM 2024)}{Saint-Petersburg,
                    Russia}{Oral Presentation}{June. 26 - 28, 2024}
  \resumeSubheading{Dynamics of Complex Networks and their Applications (DCNA
                    2023)}{Kaliningrad, Russia}{Poster Presentation}{Sep. 18 -
                    20, 2023}
  \resumeSubheading{Game Theory and Management (GTM 2023)}{Saint-Petersburg,
                    Russia}{Oral Presentation}{June. 28 - 30, 2023}
  \resumeSubheading{Control Processes and Stability 2022}{Saint-Petersburg,
                    Russia}{Oral Presentation}{Apr. 4 - 8, 2022}
  \resumeSubheading{Control Processes and Stability 2021}{Saint-Petersburg,
                    Russia}{Oral Presentation}{Apr. 5 - 9, 2021}
  \resumeSubheading{The Computing Conference 2017}{Hangzhou,
                    China}{Visitor}{Oct. 11 - 14, 2017}
  \resumeSubheading{Yiban Developer Conference 2017}{Shanghai,
                    China}{Developer}{Aug. 2017}
  \resumeSubheading{Language \& Intelligence Summit 2017}{Beijing,
                    China}{Visitor}{July 23, 2017}
\end{resumeSubHeadingList}

\section{荣誉与奖项}
\setlength{\tabcolsep}{6.0pt}
\renewcommand{\arraystretch}{1.1}
\fontsize{8.5}{11}\selectfont
\awardsTable{
  \awardsTableRow{公共开发之星}{-}{华为公司}{公共开发部}{2024年10月} \\
  \awardsTableRow{公共开发之星}{-}{华为公司}{公共开发部}{2023年06月} \\
  \awardsTableRow{公共开发之星}{-}{华为公司}{公共开发部}{2022年10月} \\
  \awardsTableRow{总裁奖(团队)}{-}{华为公司}{MAE-M}{2021年12月} \\
  \awardsTableRow{优秀毕业证书}{-}{学校}{圣彼得堡国立大学}{2021年06月} \\
  \awardsTableRow{优秀毕业论文(设计)}{-}{学校}{延安大学}{2019年06月} \\
  \awardsTableRow{优秀毕业生}{-}{学校}{延安大学}{2019年06月} \\
  \awardsTableRow{优秀学生奖学金}{-}{学校}{延安大学}{2018年12月} \\
  \awardsTableRow{中国软件杯大赛}{三等奖}{全国}{中国软件杯组委会}{2018年10月} \\
  \awardsTableRow{计算机设计大赛}{三等奖}{西北赛区}{西北大学(中国)}{2018年05月} \\
  \awardsTableRow{数学建模竞赛}{二等奖}{陕西省}{中国工业与应用数学学会}{2017年12月} \\
  \awardsTableRow{数学竞赛}{三等奖}{陕西省}{中国数学会}{2016年11月} \\ \hline }

\end{document}
